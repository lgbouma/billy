%%%%%%%%%%%%%%%%%%%%%%%%%%%%%%%%%%%%%%%%%%%%%%%%%%%%%%%%%%%%%%%%%%%%%%%%%%%%%%%

\documentclass[12pt,twocolumn,tighten]{aastex62}
%\documentclass[12pt,twocolumn,tighten,trackchanges]{aastex62}
\usepackage{amsmath,amstext,amssymb}
\usepackage[T1]{fontenc}
\usepackage{apjfonts}
\usepackage[figure,figure*]{hypcap}
\usepackage{graphics,graphicx}
\usepackage{hyperref}
\usepackage{natbib}
\usepackage[caption=false]{subfig} % for subfloat

\renewcommand*{\sectionautorefname}{Section} %for \autoref
\renewcommand*{\subsectionautorefname}{Section} %for \autoref

\newcommand{\ptfo}{PTFO$\,$8-8695}
\newcommand{\ptfob}{PTFO$\,$8-8695b}

%% Reintroduced the \received and \accepted commands from AASTeX v5.2.
%% Add "Submitted to " argument.
\received{\today}
\revised{---}
\accepted{---}
\submitjournal{AAS journals.}
\shorttitle{Seeing Double: PTFO$\,$8-8695}

%\objectname{PTFO 8-8695}
%\objectname{PTFO 8-8695b}
%\objectname{CVSO 30}

\begin{document}

\defcitealias{bouma_wasp4b_2019}{B19}

%\title{Against the Planetary Interpretation of PTFO$\,$8-8695b}
% \title{Seeing Double: Still Against the Planetary Interpretation of
% PTFO$\,$8-8695b}
% \title{Seeing Double: TESS and Gaia Show That PTFO$\,$8-8695b is Unlikely a
% Planet}
%\title{Seeing Double: Indications From {\it TESS} and {\it Gaia} That
%PTFO$\,$8-8695b is Not a Planet}
\title{Seeing Double: Transient Dips and Photometric Binarity in
PTFO$\,$8-8695}

\correspondingauthor{L. G. Bouma}
\email{luke@astro.princeton.edu}

%
% key authors:
%
\author[0000-0002-0514-5538]{L. G. Bouma}
\affiliation{ Department of Astrophysical Sciences, Princeton
University, 4 Ivy Lane, Princeton, NJ 08540, USA}
%
\author[0000-0002-4265-047X]{J. N. Winn}
\affiliation{ Department of Astrophysical Sciences, Princeton
University, 4 Ivy Lane, Princeton, NJ 08540, USA}

\begin{abstract}
  PTFO$\,$8-8695b is a candidate hot Jupiter in the 7--10 million year
  old Orion-OB1a cluster. We inspected data from TESS and Gaia to
  clarify whether it is truly a planet.  The Gaia data show that
  PTFO$\,$8-8695 is a photometric binary with respect to members of
  its kinematic group.  The TESS lightcurve shows that the dominant
  variability in this system is a sinusoid with a ``long'' period
  $P_{\rm \ell}=11.96\,$hr, presumably caused by stellar rotation.
  Also present is a complex signal, previously identified as the
  planet candidate, that repeats with a ``short'' period $P_{\rm s}=
  10.74\,$hr.  The two signals beat every 4.48 days.  Although there
  is a dip in the short-period signal, ground-based photometry from
  the past decade shows that the orbital phase of the dip seems to
  have instantaneously jumped, at least once, and perhaps twice.
  Planets do not ``jump'' in orbital phase.  Given the evidence, we
  believe that PTFO$\,$8-8695 is a binary M dwarf in which one star
  shows the long rotation signal, and the other star shows
  ``transient dipping'' that has been seen in a few other young
  weak-lined T Tauri stars.  The origin of these transient dips is
  still undetermined, but our preferred explanation is eclipses of
  clouds of gas or dust at the Keplerian co-rotation radius.
\end{abstract}

\keywords{
	Exoplanet evolution (491),
  Pre-main sequence stars (1290),
	Stellar ages (1581),
	Stellar rotation (1629),
	Variable stars (1761),
  Low mass stars (2050)
}

%%%%%%%%%%%%%%%%%%%%%%%%%%%%%%%%%%%%%%%%%%%%%%%%%%%%%%%%%%%%%%%%%%%%%%%%%%%%%%%

\section{Introduction}
If \ptfob\ were a planet, it would be exceptional.  Transiting a
sub-10$\,$Myr old weak-lined T Tauri M dwarf in Orion, it would be the
youngest hot Jupiter known \citep{van_eyken_ptf_2012}.  Its orbital
period of only 10.7 hours would also give it the shortest period of
any known hot Jupiter.  With such a short period, it would almost
certainly have filled its Roche lobe, and would be actively losing
mass to its host star.  Not only that, but the rapidly rotating host
star would also be oblate and gravity darkened, and so the planet's
orbit would likely precess into and out of transitability
\citep{barnes_measurement_2013,ciardi_followup_2015,kamiaka_revisiting_2015}. 

Other lines of evidence would imply further planetary ``firsts'' for
this planet candidate.  One first would be that its transits are about
three times deeper in optical bandpasses ({\it e.g.,} $g$-band) than
in the near-infrared ({\it e.g.}, $z$-band)
\citep{onitsuka_multicolor_2017,tanimoto_evidence_2020}.  A cloud-free
hydrogen-dominated planetary atmosphere cannot explain such a
wavelength dependence.  The planet might therefore be surrounded by a
dust cloud \citep{tanimoto_evidence_2020}.  

Another first could be the direct detection of H$\alpha$ emission from
the planet itself \citep{johnskrull_h_2016}.  While the stellar
chromosphere emits in H$\alpha$, it seems that there is an additional
excess H$\alpha$ emission that could be in phase with the planetary
orbit.  The average velocity width of the excess H$\alpha$ emission is
87$\,$km$\,$s$^{-1}$, and its equivalent width is 70-80\% that of the
stellar chromosphere \citep{johnskrull_h_2016}.  The proposed
explanation is that outflowing mass from the planet may explain this
excess emission as well \citep{johnskrull_h_2016}.

There are perhaps a few challenges to the planetary interpretation (if
these ``features'' are not already seen as such).  They include that
the planet does not seem to emit infrared radiation in occulation, at
least anywhere near the expected amplitude \citep{yu_tests_2015}.  In
addition, despite measurement attempts by multiple investigators, it
does not seem to show the Rossiter effect at the amplitude expected
given the rapid stellar rotation and large planet size
\citep{yu_tests_2015,ciardi_followup_2015}.  Finally, detailed
modelling of the ``precession + gravity darkening'' transits has shown
that the necessary degree of gravity darkening is too great, given the
spectroscopically observed equatorial velocity
\citep{howarth_reappraisal_2016}.  Additionally, as the
gravity-darkened star precessed about its rotation axis, it would need
to show photometric variability that has not been observed.  While the
planetary interpetation clearly faces challenges, alternative
explanations do as well.  High-latitude accretion hotspots might
produce the observed H$\alpha$ variability, and require fine-tuning
produce dips of the approach duration. Furthermore, \ptfo\ does not
have an infrared (IR) excess associated with the presence of an inner
disk \citep[{\it e.g.},][Figure~18]{yu_tests_2015}.  Low-latitude
starspots, hot or cold, struggle to produce the necessary dip
durations.

Alternatively, between 0.1\% and 1\% of rapidly rotating low-mass
stars in $\mathcal{O}$(10)$\,$Myr old associations show narrow dips in
phase with strong stellar rotation signals \citep{rebull_usco_2018}.
The dips can persist over months, but their depths often vary, and
sometimes change immediately after stellar flares.  The explanation
proposed by \citet{stauffer_orbiting_2017} and
\citet{david_transient_2017} is that a circumstellar cloud of dust or
gas might be orbiting near the Keplerian co-rotation radius.  For this
explanation to be viable for \ptfo, a clear determination of the
stellar rotation period is necessary.  To date this has not been
possible
\citep{van_eyken_ptf_2012,koen_multicolour_2015,raetz_yeti_2016}.

We begin in Section~\ref{sec:observations} by describing newly
available observations from TESS
\citep{ricker_transiting_2015} and Gaia
\citep{gaia_collaboration_gaia_2018}.  The TESS lightcurve shows two
distinct signals, which we extract and analyze in
Section~\ref{sec:tess}.  A long-period sinusoid repeats every 12.0
hours, and is probably stellar rotation.  A short-period dip with an additional
complex modulation repeats every 10.7 hours.  Analyzing the Gaia
data in Section~\ref{sec:gaia}, we show that relative to its kinematic
group, \ptfo\ is a photometric binary.  We collect and discuss the
puzzle pieces in Section~\ref{sec:discussion}.  The orbital phase of
the dip seems to have instantaneously changed over the past decade.
In addition, a number of other young stars show lightcurve
morphologies similar to the short-period signal.  We therefore argue that PTFO$\,$8-8695 is a binary M
dwarf in which one star shows a rotation signal, and the other 
shows ``transient dipping'' caused by eclipses of
material at the Keplerian co-rotation radius.
Section~\ref{sec:conclusions} summarizes our main points.


\section{The Data}
\label{sec:observations}

\begin{figure*}[t!]
	\begin{center}
		\leavevmode
		\includegraphics[width=1\textwidth]{f1.pdf}
	\end{center}
	\vspace{-0.7cm}
	\caption{
		{\bf TESS lightcurve of \ptfo\ (Sector 6, Orbit 19).}
    {\it Top}: ``Raw'' \texttt{PDCSAP} mean-subtracted relative flux
    versus time. The beat period of 4.48 days is visible by eye.  The
    preferred model plotted underneath the data includes 2 harmonics
    at the long period $P_{\rm \ell}$, plus 2 harmonics and a transit
    at the short period $P_{\rm s}$.
		{\it Upper middle}: Long-period signal, equal to the raw signal
		minus the short-period signal.
		{\it Lower middle}: Short-period signal, equal to the raw signal
		minus the long-period signal.
		{\it Bottom}: residual.  The data are binned from 2 to 10 minute
		cadence as a convenience for plotting and fitting.
		\label{fig:splitsignal}
	}
\end{figure*}

\begin{figure*}[hbtp]
	\begin{center}
		\leavevmode
		\includegraphics[width=1\textwidth]{f2.pdf}
	\end{center}
	\vspace{-0.7cm}
  \caption{ {\bf TESS lightcurve of \ptfo\ (Sector 6, Orbit 20).}
		Panels are as in Figure~\ref{fig:splitsignal}.
		\label{fig:splitsignalii}
	}
\end{figure*}

\subsection{TESS Observations}

\ptfo\ was observed by TESS with Camera 1, CCD 1, from December 15,
2018 to January 6, 2019, during the sixth sector of science operations
\citep{ricker_transiting_2015}.  The star was designated TIC 264461976
in the TESS Input Catalog \citep{stassun_TIC_2018,stassun_TIC8_2019}.
The pixel data for an $11\times11$ array surrounding \ptfo\ were
averaged into 2-minute stacks by the onboard computer.  Each
2048$\times$2048 image from the CCD was also averaged into 30-minute
stacks, and saved as a ``full frame image'' (FFI).

The 2-minute stacks for \ptfo\ were reduced to lightcurves by the
Science Processing Operations Center (SPOC) at NASA
Ames~\citep{jenkins_tess_2016}.  Our main analysis used the resulting
Presearch Data Conditioning (PDC) lightcurve.  The PDC lightcurve
aperture used pixels chosen to maximize the SNR of the total flux of
the target \citep{smith_kepler_apertures_2017}.  Non-astrophysical
variability was removed through the methods discussed by
\citet{smith_kepler_PDC_2017}.

As an independent check on the shorter cadence SPOC light-curve, we
separately processed the 30-minute image stacks as part of the Cluster
Difference Imaging Photometric Survey (CDIPS;
\citealt{bouma_cluster_2019}).  Our CDIPS lightcurve  of choice used a circular
aperture with radius 1 pixel.

To clean the data, we removed all points with non-zero quality flags
\citep[{\it e.g.},][]{tess_data_product_description_2018}.  We also
masked out the first and last 6 hours of each orbit, since there is
often systematic red noise during those times.  Both the CDIPS and PDC
lightcurves showed a clear discontinuous ``jump'' in the last few days
of orbit 20, which seemed likely to be an instrumental systematic.  We
correspondingly masked out times from BJD 2458488.3 until the end of
the orbit.  The PDC lightcurve initially had 15{,}678 points.  The
quality-flag cut removed 854 points; masking the orbit edges removed an
additional 716; removing the final few days of orbit 20 removed an
additional 1079.  After cleaning, 83\% of the initial flux
measurements remained.

We normalized these points by dividing out the median flux. We opted
to then subtract by unity to simplify subsequent interpretation.  Many
of these and subsequent processing steps were performed using
\texttt{astrobase}~\citep{bhatti_astrobase_2018}. 


\subsection{Gaia Observations}

\subsubsection{Astrometry of \ptfo}

Between July 25, 2014 and May 23, 2016, Gaia measured about 300
billion centroid positions of 1{.}6 billion stars
\citep{gaia_collaboration_gaia_2016,lindegren_gaiasoln_2018,gaia_collaboration_gaia_2018}.
In the Gaia second data release (DR2), these CCD observations were
then used to estimate positions, proper motions, and parallax for the
brighest 1{.}3 billion stars, including \ptfo
\citep{lindegren_gaiasoln_2018}.  There were 121 ``good'' observations
of \ptfo, that is observations that were not strongly down-weighted in
its astrometric solution.  \ptfo\ was assigned a Gaia DR2 identifier
of 3222255959210123904.  It photometric brightness was measured using
selected bands ($G$, $Rp$, and $Bp$) of the Gaia Radial Velocity
Spectrometer \citep{cropper_gaia_2018,evans_gaia_2018}.  We accessed
the pipeline parameters for \ptfo\ using the Gaia
archive\footnote{\url{gea.esac.esa.int/archive/}}.

The majority of Gaia's derived parameters for \ptfo\ agreed with
expectation from former studies
\citep{briceno_cida_2005,van_eyken_ptf_2012}.  One novelty however was
that Gaia DR2 detected a significant ``astrometric excess'', at a
level of 10.3$\sigma$.  
We comment on the significance and interpretation of this
excess in Section~\ref{sec:gaia}.


\subsubsection{Cluster Membership}



The Orion molecular cloud complex has numerous subgroups, with ages
spanning 1 to 15$\,$Myr (CITE).
\ptfo\ has been known to be a member of the Orion OB1a association
since at least CITE (Briceno 2005).
Its relation to the broader Orion complex has been further explored by
CITEX, CITEY, and CITEZ (Briceno 05, 08, 18, whomever Van Eyken cites,
and Kounkel 18).

In describing the cluster membership of \ptfo, we follow the notation
and results of \citet{kounkel_apogee2_2018}.
\citet{kounkel_apogee2_2018} combined astrometric data from Gaia DR2
with spectroscopic data from APOGEE-2 (CITE).  They then performed a
hierarchical clustering on the six dimensional position and velocity
information to identify subgroups within the Orion complex.  From
smallest to largest groups, \ptfo\ was identified as being a member of
the following nested subgroups:
\begin{equation}
  %{\rm PTFO\,8\text{-}8695}
  % \in
  {\rm 25\,Ori\text{-}1}
  \subset {\rm 25\,Ori}
  \subset {\rm Orion\ OB1a}
  \subset {\rm Orion\ D},
\end{equation}
where from set-notation, `$\subset$' denotes ``is a proper subset of''.

While all members of the Orion complex are young relative to the
field, the internal age dispersion between different subgroups is
measurable.  The Orion Nebula Cluster (M42) is a site of ongoing
star-formation, and its stars are 1-3$\,$Myr old (CITE).
25$\,$Ori\-1, by contrast, is XX-XX$\,$Myr old (CITE).  These details
are essential when assessing any evidence for photometric binarity in
\ptfo, because there is a degeneracy between stellar luminosity and
age for stars on the pre-main-sequence.  Having a clean sample of
tightly spatially and kinematically associated stars is essential to
minimize contamination not just from the field, but from older and
younger members of the Orion complex itself.


\section{TESS Analysis}
\label{sec:tess}

\begin{figure*}[t]
	\begin{center}
		\leavevmode
		\includegraphics[width=0.99\textwidth]{f3.pdf}
	\end{center}
	\vspace{-0.7cm}
	\caption{ {\bf Phase-folded long and short-period signals.}
		{\it Top}: Long-period signal, as in Figure~\ref{fig:splitsignal}.
		{\it Bottom}: Short-period signal. The reference phase is set to the
		``planetary'' dip.  Gray points are the 10 minute cadence
		\texttt{PDCSAP} flux.  Black points are binned to 100 points per
		period.
		\label{fig:phasefold}
	}
\end{figure*}

\subsection{Inspection}

Our initial inspection of the lightcurve, in both its 2-minute PDCSAP
and 30-minute FFI forms, showed a strong sinusoidal beat signal
(Figure~\ref{fig:splitsignal}, top panel).

As a precursor to more detailed analysis, we calculated generalized
Lomb-Scargle periodograms using \texttt{astrobase}
\citep{lomb_1976,scargle_studies_1982,vanderplas_periodograms_2015,bhatti_astrobase_2018}.
The two largest peaks in the Lomb-Scargle periodogram of the
lightcurve were clearly separated at a ``short'' period $P_{\rm s}
\approx 0.448\,{\rm days}$ and a ``long'' period $P_{\rm \ell} \approx
0.499\,{\rm days}$.  The $P_{\rm \ell}$ peak had the greater power of
the two.  Smaller harmonics from each of these two dominants peaks
were also present.

% 0.14 = A1+A2
% 0.06 = A1-A2
% 2A1 = 0.20
% -> A1 = 0.1
% 2A2 = 0.08
% -> A2 = 0.04

The peak-to-peak amplitude at maximum, when the two signals
constructively interfere, is about 14\%.  At minimum, the peak-to-peak
amplitude is about 6\%.  Assuming the signals are just two sinusoids,
algebra tells us that the peak-to-peak amplitudes should therefore be
10\% for the long-period signal, and 4\% for the short-period signal.
These order-of-magnitude numbers will turn out to be roughly correct.

Initial signal-processing experiments fitting out splines or sinusoids
showed that after subtracting out the long-period signal, the
short-period signal dominated the periodogram, and vice-versa.
However it quickly became clear that it would be beneficial to
simultaneously model the signals separately, in order to preserve the
power at each frequency.




\subsection{Lightcurve Model}

We opted to model the lightcurve as a linear combination of Fourier
harmonics at the short and long periods, plus a transit at the short
period.  Symbolically, the total flux $f$ is given as
\begin{equation}
  f = f_{\rm s} + f_{\rm \ell}
  = f_{\rm transit,s} + f_{\rm Fourier,s} + f_{\rm Fourier,\ell},
\end{equation}
where $f_{\rm s}$ is the relative flux at the short period, and
$f_{\rm \ell}$ is the flux at the long period.  Writing out the
Fourier terms,
\begin{align}
  f = &f_{\rm transit,s} + \sum_{n=1}^{N} A_n \sin(n\omega_{\rm s}t)
  + \sum_{n=1}^{N} B_n \cos(n\omega_{\rm s}t)\\
  &+ \sum_{m=1}^{M} A_m \sin(m[\omega_{\rm \ell}t+\phi_{\rm \ell}])
  + \sum_{m=1}^{M} B_m \cos(m[\omega_{\rm \ell}t+\phi_{\rm \ell}]), \nonumber
\end{align}
for $N$ and $M$ the total number of harmonics at the short and long
periods, respectively, $A_i$ and $B_i$ the amplitudes for each
harmonic term (potentially negative), and $\omega_i = 2\pi / P_i$ the
angular frequency for $i$ the short or long period index.  We fixed
the ``phase-offset'' for the short period signal to be zero, and let
the reference time for the long period signal float by introducing
$\phi_{\rm \ell}$.  Since we did not a priori know how many harmonics
would be appropriate, we considered a number of different choices for
$N$ and $M$, and used the Bayesian information criterion to choose the
appropriate model (Table~\ref{tab:modelcompare}).

As an example, one possible model could be a transit, plus $N=2$
harmonics of sines and cosines at the short period, plus $M=1$
harmonics at the long period.  In this case, the free parameters would
be as follows.  For the transit, we would fit for the impact
parameter, the planet-to-star radius ratio, two quadratic limb
darkening parameters, the planet orbital period (equal to the short
period), the reference time for the transit, and the mean flux.  There
would be $2N=4$ additional Fourier amplitudes at the short period,
plus $2M=2$ Fourier amplitudes at the long period, and well as the
long period itself and its phase.  For this case, we therefore fitted
14 free parameters.

We implemented and fitted the models using \texttt{PyMC3}, which is
built on \texttt{theano}
\citep{salvatier_2016_PyMC3,exoplanet:theano}.  For the Fourier terms,
we used the default math operators.  For the exoplanet transit, we
used the model and derivatives implemented in \texttt{exoplanet}
\citep{exoplanet:exoplanet}.  Our priors are listed in
Table~\ref{tab:posterior}.  To speed up the fitting, we binned the
cleaned 2 minute lightcurves to 10 minute bins.  We correspondingly
scaled the uncertainties in the flux measurements by a factor of
$\sqrt{5}$.  Before sampling, we initialized each model to the maximum
a posteriori (MAP) solution.  We then sampled using \texttt{PyMC3}'s
gradient-based No-U-Turn Sampler \citep{hoffman_no-u-turn_2014}, and
used $\hat{R}$ as our convergence diagnostic
\citep{gelman_inference_1992}.
We tested our ability to successfully recover injected parameters
using synthetic data, before switching to the actual \ptfo\
lightcurves.


\subsection{Fitting Results}

We considered nine models, with the number of harmonics per frequency
$N$ and $M$ ranging from one to three.  To select our preferred model,
we used the Bayesian information criterion
(Table~\ref{tab:modelcompare}).  The model with the lowest BIC had
two harmonics at the short 10.74$\,$hr period, and two harmonics
at the long 11.96$\,$hr period.  All of the models have reduced
$\chi^2$ ranging between 1.37 and 1.51, which suggests a plausible
though imperfect agreement between the data and models.

To explore where each model succeeded and failed, we split the raw
signal into its respective components (Figures~\ref{fig:splitsignal}
and~\ref{fig:splitsignalii}).  We also examined the phase-folded
signals (Figure~\ref{fig:phasefold}).  

In every model, the variability at the long period is a simple
sinusoid with peak-to-peak amplitude $\approx$10\%.  The variability
at the short period is always more complex.  A dip of depth
$\approx$1.2\%, fit in our model as a transit, lasts $\approx$0.75
hours.  Superposed on the dip is a complex signal with peak-to-peak
amplitude of about 4\%, which peaks near phase 0.25, and reaches
minimum brightness between phases -0.5 and -0.25.

Outside of the primary dip, the short-period signal is relatively
smooth, at least from phases 0 to 0.5.  However the short-period
signal is asymmetric.  The flux from phases -0.5 to 0 shows what could
be a discontinuous jump, shortly after reaching minimum.  This jump
was visible in each of the nine models we considered.

The periodogram of the final residual (Figure~\ref{fig:splitsignal}
bottom row) shows a weakly significant, poorly resolved peak at
$\approx$8 days, consistent with the visual impression in the time
domain that there could be a weak long-period signal present.


\subsection{Blend considerations}
\label{subsec:blend}

\begin{figure}[t]
	\begin{center}
		\leavevmode
		\includegraphics[width=0.45\textwidth]{f4.pdf}
	\end{center}
	\vspace{-0.7cm}
	\caption{ {\bf Scene used for blend analysis.}
		{\it Top:} Mean TESS image of \ptfo\ over Sector~6, with a
		log-stretch.  The position of \ptfo\ is shown with a yellow
		star.  Neighbors with $T<17$ are shown with orange crosses.  The
		apertures used to measure the background and target star flux are
		shown with \texttt{X} and \texttt{/} hatches, respectively.
		{\it Bottom:} Digitized Sky Survey $R$-band image of the same
		field, with a linear stretch. The circles show apertures of radii
		1, 1.5, and 2.25 pixels used in part of our blend analysis.  The
		pixel level TESS data show that ``Star A''  does not contribute
		variability at either of the two observed periods (see
		Section~\ref{subsec:blend}).
		\label{fig:scene}
	}
\end{figure}


The TESS pixels are $\approx21$'' per side, and so before making an
interptations, we need to consider whether light from neighboring
stars could have affected the photometry.  The scene is shown in
Figure~\ref{fig:scene}.  The pixels used to measure the background
level in the SPOC lightcuirve are indicated with an `\texttt{X}'
hatch, and the pixels used in the final lightcurve aperture are shown
with the `\texttt{/}' hatch.

The target star, \ptfo\ (TIC 264461976), has a $T$-band magnitude
of 14.0, and its position is shown with a star.  The other (unlabeled)
star inside the target aperture, TIC 264461979, has $T=16.8$ and so
cannot contribute a signal with relative amplitude 10\%.  The only
neighbor that is sufficiently close and bright that its light might
contaminate the target star is TIC 264461980, with $T=14.8$, which we
denote ``Star A''.  Star A is 23.6'' NW of our target, and based on
the magnitude difference could contribute up to 48\% the flux of our
target star, \ptfo.  

Because \ptfo\ has previously been identified to have periodicity
consistent with our measurement of $P_{\rm s}$, our main concern
regarding blending is the degree to which we can be certain that the
long-period signal at $P_{\rm \ell}$ also originates from \ptfo.
We took two approaches towards determining the source of the
long-period signal.

First, we examined the CDIPS full frame image lightcurves of the
target, which are available on MAST \citep{bouma_cluster_2019}.  The
maximal peak-to-peak beat amplitude is consistently $\approx$10\%
across apertures of radii 1, 1.5, and 2.25 pixels.  If Star A were the
source of the long-period variability, we would expect the peak
variability amplitude to be smallest in the 1 pixel aperture, based on
the separation of the sources (Figure~\ref{fig:scene}, bottom).  From
this test alone, it seems unlikely that Star A is the source of the
long-period signal.

Second, we examined the lightcurve of each pixel in the scene
individually.  We opted to use the interactive tools implemented in
\texttt{lightkurve} \citep{lightkurve_2018}.  If Star A were the
source of the long-period variability, we would expect the pixels
nearest to Star A to show a sinusoidal signal with amplitude exceeding
$10\%$.  We find no evidence for this being the case.  The pixel
directly below Star A does not clearly show the sinusoidal
variability, and the peak-to-peak variability in that pixel is
$\lesssim 8\%$.  In contrast, the south-easternmost pixel within
\ptfo's aperture (the pixel furthest from Star A that was used in the
optimal aperture) shows the $P_{\rm \ell}$ sinusoidal variability
signal at $\approx 10\%$ amplitude.

As there is no evidence in favor of a blend scenario, we conclude that
both the $P_{\rm s}$ and $P_{\rm \ell}$ signals originate from \ptfo,
at least within the resolution of the Gaia-DR2 source catalog.
However, as we shall see, \ptfo itself could still be a binary.


\section{Gaia Analysis}
\label{sec:gaia}

% \begin{figure*}[t]
% 	\begin{center}
% 		\leavevmode
% 		\includegraphics[width=0.9\textwidth]{f6.png}
% 	\end{center}
% 	\vspace{-0.7cm}
% 	\caption{ {\bf Positions, kinematics, and HR diagram of \ptfo.}
%   Members of the 25$\,$Ori-1 subgroup are shown with black circles,
%   and were identified by \citet{kounkel_apogee2_2018} through
%   clustering on Gaia-DR2 and APOGEE data.  The ``neighborhood'' (gray
%   circles) is defined as the group of at most $10^4$ randomly selected
%   non-member stars within 5 standard deviations of the mean right
%   ascension, declination, and parallax.
% 	\label{fig:gaia}
% 	}
% \end{figure*}
\begin{figure*}[t]
	\begin{center}
		\leavevmode
		%\includegraphics[width=0.8\textwidth]{f5a.pdf}
    \subfloat{
        \includegraphics[width=0.7\textwidth]{f5a}
    }

	\vspace{-0.7cm}
    \subfloat{
        \includegraphics[width=0.7\textwidth]{f5b}
    }
	\end{center}
	\vspace{-0.7cm}
  \caption{ {\bf Evidence for binarity in \ptfo}.
  {\it Top: Hertzsprung-Russell diagram of \ptfo\ and late-type
  members of 25$\,$Ori-1.}
  Members of the 25$\,$Ori-1 group (black circles) were identified by
  \citet{kounkel_apogee2_2018} through clustering on six-dimensional
  Gaia-DR2 and APOGEE-2 data.  The ``neighborhood'' (gray
  circles) is non-member
  stars within 5 standard deviations of the mean 25$\,$Ori-1 right
  ascension, declination, and parallax.  It contains members of the
  Orion complex with its full spread of ages, in addition to field
  interlopers.  $G$ denotes Gaia broadband magnitudes, $Bp$ Gaia blue, $Rp$
  Gaia red, and $\omega_{\rm as}$ the parallax in arcseconds.  The
  $x$-axis limits have been set to show only K and M dwarf members, to
  accentuate \ptfo's separation from the single-star sequence.
  {\it Bottom: Astrometric goodness-of-fit versus $Rp$ magnitude
  for 25$\,$Ori-1 members}.
  The single-source astrometric model provides a poor fit, which
  could be due to stellar variability or binarity.
	\label{fig:gaia}
	}
\end{figure*}

%FIXME: summarize existant AO constraints here too
\citet{schmidt_direct_2016}
\citet{lee_evidence_2018}
% \citet{czesla_xray_2019}

\subsection{Photometric Binarity}
To assess the cluster membership and potential binarity of \ptfo, we
needed to identify stars with which it was and was not coeval.  The
simplest way to do this---clustering based on six-dimensional position
and kinematic information---had already been done by
\citet{kounkel_apogee2_2018}.  For simplicity, we considered the
members they identified brighter than $G_{\rm Rp}$ of 16.  This
yielded 149 stars in 25$\,$Ori-1, mostly M dwarfs.
\citet{kounkel_apogee2_2018} identified seven other smaller groups in
the Orion complex near the Be star 25$\,$Ori. These groups received
higher numbers, {\it e.g.}, 25$\,$Ori-2.

To define a set of non-member stars that nonetheless had comparable
selection functions, we defined a reference ``neighborhood'' as the
group of at most $10^4$ randomly selected non-member stars within 5
standard deviations of the mean 25$\,$Ori-1 right ascension,
declination, and parallax.  We queried these stars using the
\texttt{astroquery} package, which provides a convenient interface to
the Gaia archive (CITE, CITE).  This yielded 1{,}819 neighbors.  While
some of these stars may indeed be members of the Orion complex, or
even of 25$\,$Ori-1, enforcing this cut on positions and parallaxes
ensures that we are querying stars with comparable amounts of
interstellar reddening.

We examined the resulting five-dimensional right ascension,
declination, proper motions, and parallaxes.  The first point we noted
was that 25$\,$Ori-1 was a clearly defined over-density in each
dimension, so the cluster exists, and is different from the
neighborhood.  \ptfo\ was also clearly a member in each of these
projected dimensions.

Given our detection of two separate signals, whether \ptfo\ could be a
photometric binary was of great interest.  Figure~\ref{fig:gaia} shows
the HR diagram from which we assessed this issue.  The diagram shows
that \ptfo\ is $\approx$0.8 magnitudes brighter than the average
25$\,$Ori-1 star of the same color.  In other words, it is about twice
as bright.  It also seems to be on the photometric binary track of the
cluster, which has a few other stars.

The implication is that either {\it (i)} \ptfo\ is notably younger
than the kinematically identical cluster members, or {\it (ii)} \ptfo\
is a photometric binary.  Given the independent presence of two
resolved signals, we favor the interpretation that \ptfo\ is a binary
star system.


\subsection{Astrometric Binarity}

As noted in Section~\ref{sec:observations}, the Gaia DR2 solution for
\ptfo\ shows a 10.3$\sigma$ astrometric excess.  This astrometric
excess indicates the degree to which a single-source model fails to
explain the observed astrometric measurements.  Specifically, the
single-source astrometric model for yielded $\chi^2=325.2$.  There
are 121 astrometric measurements, and 5 free parameters, and therefore
116 degrees of freedom. The reduced $\chi^2$ is 2.80.  The
majority of stars with comparable brightness in Gaia do not show such
poor goodness-of-fit \citep[][Appendix A]{lindegren_gaiasoln_2018}.

Potential explanations for the poor astrometric fit include photometric
variability and unresolved stellar binarity \citep[{\it
e.g.},][]{rizzuto_ZEIT8_2018,belokurov_unresolved_2020}.
If photometric variability were the cause, we would expect comparably
faint stars in the same kinematic group of Orion to
show similar astrometric excesses, as the majority of
young stars are highly variable.

Using the same 149 members in the 25$\,$Ori-1 subgroup from
\citet{kounkel_apogee2_2018}, we calculated the reduced $chi^2$ for
each member.  The lower panel of Figure~\ref{fig:gaia} shows the
resulting values, as a function of stellar brightness.  \ptfo\ is in
the upper 90$^{\rm th}$ percentile of stars showing astrometric
excesses within the 25$\,$Ori-1 group.  Although we will have to wait
for the full release of the nominal Gaia mission to definitively
determine whether the astrometric excess is caused by stellar binarity
or photometric variability, this result suggests that stellar binarity
is the root cause, because other member stars have identical variability
characteristics but do not show such a large astrometric excess.
%FIXME: you might want to color the points by their TESS amplitude, or
%similar.
% or even better... just do a CUT on TESS amplitude. Require it to be
% like 10% or above -- so LARGER amplitudes than PTFO! That's the best
% test.




\section{Discussion}
\label{sec:discussion}

\subsection{Long period sinusoid}

The standard interpretation of sinusoidal modulations for a
pre-main-sequence M dwarf is that we are observing a stellar rotation
period of 11.96 hours.  This is the dominant signal in the system with
10\% amplitude, and there is no evidence to suggest that this signal
has any other origin.

The discovery study by \citet{van_eyken_ptf_2012} saw the same signal
({\it e.g.}, their Figure~7), but identified its alias as a
periodogram peak at $0.9985 \pm 0.0061\,$days. They ascribed it to
their observing cadence, because of its close correspondence to the
sidereal day.  While the TESS data can show significant reflected
light from the Earth \citep[{\it e.g.},][]{luger_tess_2019}, our
pixel-level analysis showed that the signal is specific to only pixels
near \ptfo, and no other pixels.  We therefore conclude that the
signal is not a systematic.

We are not the first to reach the conclusion that the long period
sinusoidal modulation is astrophysical.  A follow-up study by
\citet{koen_multicolour_2015} identified the same modes and aliases as
\citet{van_eyken_ptf_2012}, but argued that the $0.50\,{\rm d}$ signal
was astrophysical.  Using archival photometry and photometry from the
YETI global telescope network, \citet{raetz_yeti_2016} eventually came
to the conclusion that the that the $0.50\,{\rm d}$ signal was indeed
from stellar rotation.  The TESS data strongly support this
conclusion.



\subsection{Short period dip}
The dip lasts about 45 minutes, and seems to re-occur every 10.74
hours
(Figures~\ref{fig:splitsignal},~\ref{fig:splitsignalii},~\ref{fig:phasefold}).
The dip duration is roughly the same as that observed by previous
investigators \citep{van_eyken_ptf_2012,yu_tests_2015}
The dip depth is comparable to what has been observed in red visual
bands...
%TODO: and how does it compare to IR? bigger or smaller? cite relevant
%studies.


\begin{figure}[t]
	\begin{center}
		\leavevmode
		\includegraphics[width=0.5\textwidth]{f6.pdf}
	\end{center}
	\vspace{-0.7cm}
	\caption{
		{\bf Timing residuals for \ptfob\ from a decade of monitoring.}
    Black points are times of ``dips'', minus the indicated linear
    ephemeris.  The $y$-axis is given in units of phase for the
    short-period signal.  The star shows the binned TESS ephemeris.
    ``Dips'' have been observed by \citet{van_eyken_ptf_2012},
    \citet{ciardi_followup_2015}, \citet{yu_tests_2015},
    \citet{raetz_yeti_2016}, \citet{onitsuka_multicolor_2017}, and
    \citet{tanimoto_evidence_2020}.  Certain dips ({\it e.g.}, the one
    at phase 0 in mid-2019) are consistent with noise, and were likely
    reported because something was {\it expected}, rather than
    convincingly {\it observed}.  Horizontal dashed lines are drawn at
    $\pm (P_{\rm \ell} - P_{\rm s})/P_{\rm s}$, highlighting a
    possible observational bias.  The orbital phase observed by TESS
    is consistent with that of \citet{tanimoto_evidence_2020}, and
    quite different from the original phase.
		\label{fig:o_minus_c}
	}
\end{figure}

\paragraph{Epoch}
The TESS dip does not phase up where it is supposed to...
Figure~\ref{fig:o_minus_c}

\subsection{Short period out-of-dip modulation}
If there were a giant planet transiting \ptfo, it would tidally
distort the host star, and cause ellipsoidal photometric modulations.
The amplitude of the ellipsoidal distortion for a $1\,M_{\rm Jup}$
companion would be about 1400$\,$ppm
\citep{shporer_astrophysics_2017}.  This is significantly larger than
the typical ellipsopidal modulation induced by close-in giant planets
because the host star is puffy, and still on the pre-main-sequence.
For our estimate, we assumed $R_\star = 1.39 R_\odot$, and $M_\star =
0.39 M_\odot$ \citep{van_eyken_ptf_2012}. 

Our preferred model does detect a significant ellipsoidal signal,
parametrized as the ``$B_1$'' component.  The amplitude of the signal
is $0.53 \pm 0.06\%$.  Interpreted as being caused by a planet, it
would imply a minimum planet mass $M_{\rm p} \sin i$ of $3.8\,M_{\rm
Jup}$.




\section{Physical Interpretation}

%FIXME: the PTFO lightcurve looks so much worse! Why is it less dense??
\begin{figure*}[hbtp]
	\begin{center}
		\leavevmode
		\includegraphics[width=1\textwidth]{f7.pdf}
	\end{center}
	\vspace{-0.7cm}
  \caption{ {\bf \ptfo\ and its brethren.}
    Five transient and persistent flux dip stars, selected based on
    their visual similarity to the short-period signal in \ptfo, are
    as follows.
  	EPIC 204143627, 1.1250d, dips change depth too
  	EPIC 204321142  Usco flux dip, 0.476d
  	EPIC 204270520  USco flux dip, 0.512d
  	% EPIC 205046529B, 1.8358d, indicates dips can be at weird phases too
  	RIK-210 = EPIC 205483258
 	EPIC 204787516  USco flux dip or possible? EB, 0.487d
 	We found these objects through studies by \citet{stauffer_orbiting_2017}, \citet{david_transient_2017}, and \citet{rebull_usco_2018}.
		\label{fig:brethren}
	}
\end{figure*}

Given the evidence, we believe that PTFO$\,$8-8695 is a binary M dwarf
in which one star shows the ``long'' rotation signal, and the other is
showing ``transient dipping'' that has also been observed in other
young M dwarfs. 

Many other young M-dwarf photometric binaries have been observed in
{\it e.g.,} Upper Sco and the Pleiades to show multiple periods
\citep{rebull_usco_2018,stauffer_rotevol_2018}.  Typically, the
periods are both rotational modulation.

However, occasionally one or both periods can be ``scallop-shell''
variability, {\it e.g.}, EPIC 203956650 in $\rho$Oph
\citep{rebull_usco_2018}.




There are 

The gas in the disks is gone (CITE).
The upper limits on the SED from \citet{yu_tests_2015} imply X, Y, Z.
The stars are therefore presumably no longer ``magnetically locked''
to their disks.
This is consistent with the $\approx$half-day periodicities of both
signals.
In the broader context of rapidly rotating young M-dwarfs, ``magnetic
locking'' causes classical disked T-Tauri stars tend to rotate more
{\it slowly}, and almost none have periods less than two days
\citep[{\it e.g.},][]{rebull_rotation_2020}.

The main physical question is what is causing the ``transient
dipping''. This is an unsolved problem not only for \ptfo\ but also
for an entire class of young rapidly rotating M-dwarfs



The roughly half-day periods of both signals imply that 






\section{Conclusions}
\label{sec:conclusions}

\ptfo\ was previously thought to potentially host a hot Jupiter.
The TESS lightcurve of \ptfo\ showed a number of new features,
many of which seem to disfavor the hot Jupiter interpretatation.
The TESS data showed two key pieces of evidence.
\begin{enumerate}
  \item {\it Two periodic signals.} The ``long'' signal is a 10\%
      peak-to-peak sinusoidal modulation repeating every 11.96 hours.
      The ``short'' signal is a 4\% peak-to-peak complex modulation
      repeating every 10.74 hours. It is composed of a dip, plus at
      least two harmonics. The signals beat, and therefore cannot be
      an artifact linked to data processing.
  \item {\it A dip at the wrong orbital phase.} The clearest dip in
    the ``short'' signal was consistent with recent observations by
    \citet{tanimoto_evidence_2020}, and differed from the discovery
    epoch by 5.14 hours.
\end{enumerate}

The physical mechanism responsible for all these features remains a
matter of speculation.
With that said, the TESS data support new arguments against the planetary
interpretation of \ptfo.
First, if the long signal is caused by starspot modulation, and the
short signal by a transiting planet, what causes the additional
complex modulations seen at the short, ``orbital'', period?

Similarly, if the planet truly orbits every 10.74 hours, while the
star's equator spins every 11.96 hours, the situation is clearly
Darwin unstable. 



  Given the available evidence,
  PTFO$\,$8-8695 seems consistent with the
  ``transient dipping'' phenomenology observed in many young M dwarfs.
  It seems rather unlikely to be a planet.



%%%%%%%%%%%%%%%%%%%%%%%%%%%%%%%%%%%%%%%%%%%%%%%%%%%%%%%%%%%%%%%%%%%%%%%%%%%%%%%

% \acknowledgements
% %
% This paper includes data collected by the TESS mission, which are
% publicly available from the Mikulski Archive for Space Telescopes
% (MAST).
% %
% Funding for the TESS mission is provided by NASA's Science Mission
% directorate.
% %
% This work made use of NASA's Astrophysics Data System Bibliographic
% Services.
% %
% Based on observations obtained at the Gemini Observatory, which is
% operated by the Association of Universities for Research in Astronomy,
% Inc., under a cooperative agreement with the NSF on behalf of the
% Gemini partnership: the National Science Foundation (United States),
% National Research Council (Canada), CONICYT (Chile), Ministerio de
% Ciencia, Tecnolog\'{i}a e Innovaci\'{o}n Productiva (Argentina),
% Minist\'{e}rio da Ci\^{e}ncia, Tecnologia e Inova\c{c}\~{a}o (Brazil),
% and Korea Astronomy and Space Science Institute (Republic of Korea).
% %
% Observations in the paper made use of the High-Resolution Imaging
% instrument Zorro at Gemini-South. Zorro was funded by the NASA
% Exoplanet Exploration Program and built at the NASA Ames Research
% Center by Steve B. Howell, Nic Scott, Elliott P. Horch, and Emmett
% Quigley.
% %
% This research has made use of the VizieR catalogue access tool, CDS,
% Strasbourg, France. The original description of the VizieR service was
% published in A\&AS 143, 23.
% %
% This work has made use of data from the European Space Agency (ESA)
% mission {\it Gaia} (\url{https://www.cosmos.esa.int/gaia}), processed
% by the {\it Gaia} Data Processing and Analysis Consortium (DPAC,
% \url{https://www.cosmos.esa.int/web/gaia/dpac/consortium}). Funding
% for the DPAC has been provided by national institutions, in particular
% the institutions participating in the {\it Gaia} Multilateral
% Agreement.
%
% (Some of) The data presented herein were obtained at the W. M. Keck
% Observatory, which is operated as a scientific partnership among the
% California Institute of Technology, the University of California and
% the National Aeronautics and Space Administration. The Observatory was
% made possible by the generous financial support of the W. M. Keck
% Foundation.
% The authors wish to recognize and acknowledge the very significant
% cultural role and reverence that the summit of Maunakea has always had
% within the indigenous Hawaiian community.  We are most fortunate to
% have the opportunity to conduct observations from this mountain.
%
% \newline
%

\software{
  \texttt{astrobase} \citep{bhatti_astrobase_2018},
  % \texttt{astroplan} \citep{astroplan2018},
  \texttt{astropy} \citep{astropy_2018},
  \texttt{astroquery} \citep{astroquery_2018},
  % \texttt{BATMAN} \citep{kreidberg_batman_2015},
  \texttt{corner} \citep{corner_2016},
  %\texttt{emcee} \citep{foreman-mackey_emcee_2013},
  \texttt{exoplanet} \citep{exoplanet:agol19}
  \texttt{exoplanet} \citep{exoplanet:exoplanet}, and its
  dependencies \citep{exoplanet:agol19, exoplanet:kipping13, exoplanet:luger18,
  	exoplanet:theano}.
  \texttt{IPython} \citep{perez_2007},
	\texttt{lightkurve} \citep{lightkurve_2018},
  \texttt{matplotlib} \citep{hunter_matplotlib_2007}, 
  \texttt{MESA} \citep{paxton_modules_2011,paxton_modules_2013,paxton_modules_2015}
  \texttt{numpy} \citep{walt_numpy_2011}, 
  \texttt{pandas} \citep{mckinney-proc-scipy-2010},
  \texttt{PyMC3} \citep{salvatier_2016_PyMC3},
  \texttt{radvel} \citep{fulton_radvel_2018},
  % \texttt{scikit-learn} \citep{scikit-learn},
  \texttt{scipy} \citep{jones_scipy_2001}.
}


% \facilities{
% 	{\it Astrometry}:
% 	Gaia \citep{gaia_collaboration_gaia_2016,gaia_collaboration_gaia_2018}.
% 	{\it Imaging}:
% 	Gemini:South~(Zorro; \citealt{scott_nessi_2018}.
% 	{\it Spectroscopy}:
% 	Keck:I~(HIRES; \citealt{vogt_hires_1994}),
% 	Euler1.2m~(CORALIE),
% 	ESO:3.6m~(HARPS; \citealt{mayor_setting_2003}).
% 	{\it Photometry}:
% 	CTIO:1.0m (Y4KCam),
% 	Danish 1.54m Telescope,
% 	El Sauce:0.356m,
% 	Elizabeth 1.0m at SAAO,
% 	Euler1.2m (EulerCam),
% 	Magellan:Baade (MagIC),
% 	Max Planck:2.2m	(GROND; \citealt{greiner_grond7-channel_2008})
% 	NTT,
% 	SOAR (SOI),
% 	TESS \citep{ricker_transiting_2015},
% 	TRAPPIST \citep{jehin_trappist_2011},
% 	VLT:Antu (FORS2).
% }

%
% The following are entries from Table 1 that are not otherwise cited
% in the text
%
% \nocite{wilson_wasp-4b_2008}
% \nocite{gillon_improved_2009}
% \nocite{winn_transit_2009}
% \nocite{hoyer_tramos_2013}
% \nocite{dragomir_terms_2011}
% \nocite{sanchis-ojeda_starspots_2011}
% \nocite{nikolov_wasp-4b_2012}
% \nocite{ranjan_atmospheric_2014}
% \nocite{huitson_gemini_2017}

% \input{WASP-4b_transit_time_table.tex}
% \input{WASP-4b_rv_table.tex}
% \input{model_fit_table.tex}
% \input{rv_model_posterior_table.tex}
% \input{pdot_table.tex}

\clearpage
\startlongtable
\begin{deluxetable*}{lrrrrrrrr}
%
%\tabletypesize{\scriptsize}
%
\tablenum{1}
%
\tablecaption{Model Comparison.}
\label{tab:modelcompare}
%
\tablehead{
\colhead{Description} &
\colhead{$N_{\rm s}$} &
\colhead{$N_{\rm \ell}$} &
\colhead{$N_{\rm data}$} &
\colhead{$N_{\rm param}$} &
\colhead{$\chi^2$} &
\colhead{$\chi_{\rm red}^2$} &
\colhead{BIC} &
\colhead{$\Delta$BIC}
}
% pasted from
% /Users/luke/Dropbox/proj/billy/results/PTFO_8-8695_results/20200513_v0/bic_table_data.tex
% 
% Burnham and Anderson 2004.
% "Models having i ≤ 2 have substantial support (evidence), those in which 4 ≤
% i ≤ 7 have considerably less support, and models having i > 10 have
% essentially no support"
\startdata
Favored    & 3 &  2 &   2585 &      21 &  3102.4 &     1.210 &  3267.4 &     0.0 \\
% \hline
% Weakly favored &  
\hline
Disfavored  &  2 &  3 &   2585 &      21 &  3179.0 &     1.240 &  3344.0 &    76.6 \\
---         &  2 &  2 &   2585 &      19 &  3237.4 &     1.262 &  3386.7 &   119.3 \\
---         &  3 &  3 &   2585 &      23 &  3217.1 &     1.256 &  3397.9 &   130.4 \\
---         &  2 &  1 &   2585 &      17 &  3312.6 &     1.290 &  3446.1 &   178.7 \\
---         &  3 &  1 &   2585 &      19 &  3397.5 &     1.324 &  3546.8 &   279.4 \\
---         &  1 &  2 &   2585 &      17 &  4101.2 &     1.597 &  4234.8 &   967.3 \\
---         &  1 &  3 &   2585 &      19 &  4160.8 &     1.622 &  4310.1 &  1042.7 \\
---         &  1 &  1 &   2585 &      15 &  4318.4 &     1.680 &  4436.2 &  1168.8 \\
\enddata
%
\tablecomments{
	$N_{\rm s}$ and $N_{\rm \ell}$ are the number of harmonics at the short and long periods, respectively.
	$N_{\rm data}$ is the number of fitted flux measurements.
	$N_{\rm param}$ is the number of free parameters in the model.
	The Bayesian information criterion (BIC) and the difference from the maximum $\Delta {\rm BIC}$ are also listed.
}
\vspace{-1cm}
\end{deluxetable*}

% Table of best fit parameters
\startlongtable
\begin{deluxetable*}{lrrr}
%
\tablecaption{ Best-fit radial velocity model parameters. }
\label{tab:posterior}
%
\tablenum{4}
%
\tablehead{
  \colhead{Parameter} & 
  \colhead{Credible Interval} & 
  \colhead{Maximum Likelihood} & 
  \colhead{Units}
}
\startdata
%\sidehead{~~~~~\it{Modified MCMC Step Parameters}}
%  $P_{b}$ & $1.338231466\pm 2.3e-08$ & $1.338231466$ & day \\
%  $T\rm{conj}_{b}$ & $2455804.515752^{+2.5e-05}_{-2.4e-05}$ & $2455804.515752$ & BJD$_{\rm TDB}$ \\
%  $e_{b}$ & $\equiv0.0$ & $\equiv0.0$ &  \\
%  $\omega_{b}$ & $\equiv0.0$ & $\equiv0.0$ & $^\circ$ \\
%  $K_{b}$ & $242.6^{+3.6}_{-3.5}$ & $242.6$ & m$\,{\rm s}^{-1}$ \\
%\hline
\sidehead{~~~~~\it{Orbital Parameters}}
  $P_{b}$ & $1.338231466\pm 2.3e-08$ & $1.338231466$ & day \\
  $T\rm{conj}_{b}$ & $2455804.515752^{+2.5e-05}_{-2.4e-05}$ & $2455804.515752$ & BJD$_{\rm TDB}$ \\
  $e_{b}$ & $\equiv0.0$ & $\equiv0.0$ &  \\
  $\omega_{b}$ & $\equiv0.0$ & $\equiv0.0$ & $^\circ$ \\
  $K_{b}$ & $242.6^{+3.6}_{-3.5}$ & $242.6$ & m$\,{\rm s}^{-1}$ \\
%\hline
\sidehead{~~~~~\it{Other Parameters}}
  $\gamma_{\rm HIRES}$ & $36.4^{+5.8}_{-5.9}$ & $36.4$ & m$\,{\rm s}^{-1}$ \\
  $\gamma_{\rm HARPS}$ & $-69.9^{+4.2}_{-4.1}$ & $-70.1$ & m$\,{\rm s}^{-1}$ \\
  $\gamma_{\rm CORALIE}$ & $-39.9^{+5.5}_{-5.2}$ & $-40.1$ & m$\,{\rm s}^{-1}$ \\
  $\dot{\gamma}$ & $-0.0422^{+0.0028}_{-0.0027}$ & $-0.0424$ & m$\,{\rm s}^{-1}\,{\rm day}^{-1}$ \\
  $\ddot{\gamma}$ & $\equiv0.0$ & $\equiv0.0$ &  \\
  $\sigma_{\rm HIRES}$ & $10.8^{+3.7}_{-2.7}$ & $8.2$ & $\rm m\,s^{-1}$ \\
  $\sigma_{\rm HARPS}$ & $13.0^{+3.7}_{-2.6}$ & $11.5$ & $\rm m\,s^{-1}$ \\
  $\sigma_{\rm CORALIE}$ & $13.8^{+6.6}_{-6.7}$ & $12.9$ & $\rm m\,s^{-1}$ \\
\enddata
%\tablenotetext{}{ 240000 links saved}
\tablenotetext{}{
  Reference epoch for $\gamma$,$\dot{\gamma}$,$\ddot{\gamma}$: 2455470 
}
\vspace{-2.5cm}
\end{deluxetable*}

\clearpage

\bibliographystyle{yahapj}                            
\bibliography{bibliography} 


\listofchanges

\end{document}
