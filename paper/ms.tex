%%%%%%%%%%%%%%%%%%%%%%%%%%%%%%%%%%%%%%%%%%%%%%%%%%%%%%%%%%%%%%%%%%%%%%%%%%%%%%%

\documentclass[12pt,twocolumn,tighten]{aastex62}
%\documentclass[12pt,twocolumn,tighten,trackchanges]{aastex62}
\usepackage{amsmath,amstext,amssymb}
\usepackage[T1]{fontenc}
\usepackage{apjfonts}
\usepackage[figure,figure*]{hypcap}
\usepackage{graphics,graphicx}
\usepackage{hyperref}
\usepackage{natbib}

\renewcommand*{\sectionautorefname}{Section} %for \autoref
\renewcommand*{\subsectionautorefname}{Section} %for \autoref

%% Reintroduced the \received and \accepted commands from AASTeX v5.2.
%% Add "Submitted to " argument.
\received{\today}
\revised{---}
\accepted{---}
\submitjournal{AAS journals.}
\shorttitle{Against PTFO$\,$8-8695b}

\begin{document}

\defcitealias{bouma_wasp4b_2019}{B19}

\title{Against the Planetary Interpretation of PTFO$\,$8-8695b}

\correspondingauthor{L. G. Bouma}
\email{luke@astro.princeton.edu}

%
% key authors:
%
\author[0000-0002-0514-5538]{L. G. Bouma}
\affiliation{ Department of Astrophysical Sciences, Princeton
University, 4 Ivy Lane, Princeton, NJ 08540, USA}
%
\author[0000-0002-4265-047X]{J. N. Winn}
\affiliation{ Department of Astrophysical Sciences, Princeton
University, 4 Ivy Lane, Princeton, NJ 08540, USA}

%
% contributing authors: alphabetical
%
% \author[0000-0001-8638-0320]{A. W. Howard}
% \affiliation{Cahill Center for Astrophysics, California Institute of
% Technology, Pasadena, CA 91125, USA}
% %
% \author[0000-0002-2532-2853]{S. B. Howell}
% \affiliation{NASA Ames Research Center, Moffett Field, CA 94035, USA}
% %
% \author[0000-0002-0531-1073]{H. Isaacson}
% \affiliation{Astronomy Department, University of California, Berkeley,
% CA 94720, USA}
% %
% \author{H. Knutson}
% \affiliation{Division of Geological and Planetary Sciences, California
% Institute of Technology, Pasadena, CA 91125, USA}
% %
% \author[0000-0001-7233-7508]{R. A. Matson}
% \affiliation{U.S. Naval Observatory, Washington, DC 20392, USA}
% %

\begin{abstract}
  PTFO$\,$8-8695b could be the youngest, shortest-period 
  hot Jupiter known.  However it has not been shown to be a planet.
  TESS recently observed PTFO$\,$8-8695 for one month.
  The TESS lightcurve shows that the dominant variability in this
  system is a sinusoidal modulation with a ``long'' period $P_{\rm
  \ell}$
  of 11.96 hours, likely caused by stellar rotation.
  Also present is a complex signal, previously identified as the
  planet candidate, that repeats with a ``short'' period $P_{\rm s}$ of
  10.74 hours.
  The ``long'' and ``short'' signals show the expected beat every 4.48
  days.
  There is a dip in the complex, short-period signal.
  However ground-based photometry from the past decade shows that the
  orbital phase of the dip seems to have instantaneously jumped, at
  least once, and maybe twice.
  The TESS epoch of the dip is consistent with recent observations by
  Tanimoto et al., and differs from the discovery epoch by 5.14 hours.
  Planets do not ``jump'' in orbital phase.
  Given the available evidence,
  PTFO$\,$8-8695 seems consistent with the
  ``transient dipping'' phenomenology observed in many young M dwarfs.
  It seems rather unlikely to be a planet.
\end{abstract}

%TODO
\keywords{}

%%%%%%%%%%%%%%%%%%%%%%%%%%%%%%%%%%%%%%%%%%%%%%%%%%%%%%%%%%%%%%%%%%%%%%%%%%%%%%%

\section{Introduction}
If PTFO$\,$8-8695b were a planet, it would be exceptional.
A transiting hot Jupiter, orbiting a $\approx$3$\,$Myr old M dwarf
in ORION would make it the youngest hot Jupiter known.
Its orbital period of only 12 HOURS would also make it the shortest
period hot Jupiter known.

% Section~\ref{sec:observations} of this paper presents all of the
% available transit data as well as the new radial velocity and speckle
% imaging observations.  Section~\ref{sec:analysis} describes our
% analysis of the data, and our interpretation that WASP-4 is being
% pulled around by a brown dwarf or low-mass star.
% Section~\ref{sec:discussion} places this result within the context of
% orbital decay searches, and points out that line-of-sight
% accelerations will be a relatively common type of ``false positive.''
% Section~\ref{sec:conclusions} offers concluding remarks.

\section{The Data}
\label{sec:observations}

\begin{figure*}[t!]
	\begin{center}
		\leavevmode
		\includegraphics[width=1\textwidth]{f1.pdf}
	\end{center}
	\vspace{-0.7cm}
	\caption{
		{\bf TESS lightcurve of PTFO 8-8695 (Sector 6, Orbit 19).}
		{\it Top}: ``Raw'' \texttt{PDCSAP} mean-subtracted relative flux
		versus time. The beat period of 4.48 days is visible by eye.  The
		model plotted underneath the data includes 2 harmonics at the long
		period $P_{\rm \ell}$, plus 2 harmonics and a transit at the short
		period $P_{\rm s}$.
		{\it Upper middle}: Long-period signal, equal to the raw signal
		minus the short-period signal.
		{\it Lower middle}: Short-period signal, equal to the raw signal
		minus the long-period signal.
		{\it Bottom}: residual.  The data are binned from 2 to 10 minute
		cadence as a convenience for plotting and fitting.
		\label{fig:splitsignal}
	}
\end{figure*}

%\newpage
\begin{figure*}[hbtp]
	\begin{center}
		\leavevmode
		\includegraphics[width=1\textwidth]{f2.pdf}
	\end{center}
	\vspace{-0.7cm}
	\caption{ {\bf TESS lightcurve of PTFO 8-8695 (Sector 6, Orbit 20).}
		Panels are as in Figure~\ref{fig:splitsignal}.
		\label{fig:splitsignalii}
	}
\end{figure*}
%\newpage



\begin{figure*}[t]
	\begin{center}
		\leavevmode
		\includegraphics[width=0.95\textwidth]{f3.pdf}
	\end{center}
	\vspace{-0.7cm}
	\caption{ {\bf Phase-folded long and short-period signals.}
		{\it Top}: Long-period signal, as in Figure~\ref{fig:splitsignal}.
		{\it Bottom}: Short-period signal. The reference phase is set to the
		``planetary'' dip.  Gray points are the 10 minute cadence
		\texttt{PDCSAP} flux.  Black points are binned to 100 points per
		period.
		\label{fig:phasefold}
	}
\end{figure*}


\begin{figure}[t]
	\begin{center}
		\leavevmode
		\includegraphics[width=0.48\textwidth]{f4.pdf}
	\end{center}
	\vspace{-0.7cm}
	\caption{ {\bf Scene used for blend analysis.}
		{\it Top:} Mean TESS image of PTFO 8-8695 over Sector~6, with a
		log-stretch.  The position of PTFO 8-8695 is shown with a yellow
		star.  Neighbors with $T<17$ are shown with orange crosses.  The
		apertures used to measure the background and target star flux are
		shown with \texttt{X} and \texttt{/} hatches, respectively.
		{\it Bottom:} Digitized Sky Survey $R$-band image of the same
		field, with a linear stretch. The circles show apertures of radii
		1, 1.5, and 2.25 pixels used in part of our blend analysis.  The
		pixel level TESS data show that ``Star A''  does not contribute
		variability at either of the two observed periods (see
		Section~\ref{subsec:blend}).
		\label{fig:scene}
	}
\end{figure}

\begin{figure*}[t]
	\begin{center}
		\leavevmode
		\includegraphics[width=0.9\textwidth]{f5.pdf}
	\end{center}
	\vspace{-0.7cm}
	\caption{
		{\bf Timing residuals for PTFO 8-8695b from a decade of monitoring.}
		Black points are times of ``dips'', minus the indicated linear
		ephemeris.  The $y$-axis is given in units of phase for the
		short-period signal.  The star shows the binned TESS ephemeris.
		``Dips'' have been observed by \citet{van_eyken_ptf_2012},
		\citet{ciardi_follow-up_2015}, \citet{yu_tests_2015},
		\citet{raetz_yeti_2016}, \citet{onitsuka_multi-color_2017}, and
		\citet{tanimoto_evidence_2020}.  Certain dips ({\it e.g.}, the one
		at phase 0 in mid-2019) are consistent with noise, and were likely
		reported because something was {\it expected}, rather than
		convincingly {\it observed}.  Horizontal dashed lines are drawn at
		$\pm (P_{\rm \ell} - P_{\rm s})/P_{\rm s}$, highlighting what
		could be either a numerical coincidence or an observational bias.
		The orbital phase observed by TESS is consistent with that of
		\citet{tanimoto_evidence_2020}, and quite different from the
		original phase.
		\label{fig:o_minus_c}
	}
\end{figure*}

\begin{figure*}[t!]
	\begin{center}
		\leavevmode
		\includegraphics[width=0.9\textwidth]{f6_comp.png}
	\end{center}
	\vspace{-0.7cm}
	\caption{ {\bf foo.}
		bar
		\label{fig:corner}
	}
\end{figure*}

\subsection{Observations}

PTFO 8-8695 was observed by TESS with Camera 1, CCD 1, from December
15, 2018 to January 6, 2019, during the sixth sector of science
operations.  The star was designated TIC 264461976 in the TESS Input
Catalog \citep{stassun_TIC_2018,stassun_TIC8_2019}.  The pixel data
for an $11\times11$ array surrounding PTFO 8-8695 were averaged into
2-minute stacks by the onboard computer.  Each 2048$\times$2048 image
from the CCD was also averaged into 30-minute stacks, and saved as a
``full frame image'' (FFI).

The 2-minute stacks for PTFO 8-8695 were then reduced to lightcurves
by the Science Processing Operations Center (SPOC) at NASA
Ames~\citep{jenkins_tess_2016}.  Our main analysis used the resulting
Presearch Data Conditioning (PDC) lightcurve.  The PDC lightcurve
aperture used pixels chosen to maximize the SNR of the total flux of
the target \citep{smith_kepler_apertures_2017}.  Non-astrophysical
variability was removed through the methods discussed by
\citet{smith_kepler_PDC_2017}.

As an independent check on the shorter cadence SPOC light-curve, we
separately processed the 30-minute image stacks as part of the Cluster
Difference Imaging Photometric Survey (CDIPS;
\citep{bouma_cluster_2019}).  The CDIPS lightcurve used a circular
aperture with radius 1 pixel.

To clean the data, we removed all points with non-zero quality flags
\citep[{\it e.g.},][]{tess_data_product_description_2018}.  We also
masked out the first and last 6 hours of each orbit, since there is
often systematic red noise during those times.  Both the CDIPS and PDC
lightcurves showed a clear discontinuous ``jump'' in the last few days
of orbit 20, which seemed likely to be an instrumental systematic.  We
correspondingly masked out times from BJD 2458488.3 until the end of
the orbit.  The PDC lightcurve initially had 15{,}678 points.  The
quality cut removed 854 points, masking the orbit edges removed an
additional 716, and removing the final few days of orbit 20 removed an
additional 1079.  After cleaning, 83\% of the initial flux
measurements remained.

We normalized these points by dividing out the median flux. We then
subtracted by unity to simplify subsequent analysis.  Many of these
and subsequent processing steps were performed using
\texttt{astrobase}~\citep{bhatti_astrobase_2018}. 


\subsection{Visual Inspection}

Our initial inspection of the lightcurve, in both its 2-minute PDCSAP
and 30-minute FFI forms, showed a strong sinusoidal beat signal
(Figure~\ref{fig:splitsignal}, top panel).

As a precursor to more detailed analysis, we calculated generalized
Lomb-Scargle periodograms using \texttt{astrobase}
\citep{lomb_1976,scargle_studies_1982,vanderplas_periodograms_2015,bhatti_astrobase_2018}.
The two largest peaks in the Lomb-Scargle periodogram of the
lightcurve were clearly separated at a ``short'' period $P_{\rm s}
\approx 0.448\,{\rm days}$ and a ``long'' period $P_{\rm \ell} \approx
0.499\,{\rm days}$.  The $P_{\rm \ell}$ peak had the greater power of
the two.  Smaller harmonics from each of these two dominants peaks
were also present.

% 0.14 = A1+A2
% 0.06 = A1-A2
% 2A1 = 0.20
% -> A1 = 0.1
% 2A2 = 0.08
% -> A2 = 0.04

The peak-to-peak amplitude at maximum, when the two signals
constructively interfere, is about 14\%.  At minimum, the peak-to-peak
amplitude is about 6\%.  Assuming the signals are just two sinusoids,
algebra tells us that the peak-to-peak amplitudes should therefore be
10\% for the long-period signal, and 4\% for the short-period signal.
These order-of-magnitude numbers will turn out to be roughly correct.

Initial signal-processing experiments fitting out splines or sinusoids
showed that after subtracting out the long-period signal, the
short-period signal dominated the periodogram, and vice-versa.
However it quickly became clear that it would be beneficial to
simultaneously model the signals separately, in order to preserve the
power at each frequency.



\section{The Model}
\label{sec:analysis}

\subsection{Model Description}

We opted to model the lightcurve as a linear combination of Fourier
harmonics at the short and long periods, plus a transit at the short
period.  Symbolically, the total flux $f$ is given as
\begin{equation}
  f = f_{\rm s} + f_{\rm \ell}
  = f_{\rm transit,s} + f_{\rm Fourier,s} + f_{\rm Fourier,\ell},
\end{equation}
where $f_{\rm s}$ is the relative flux at the short period, and
$f_{\rm \ell}$ is the flux at the long period.  Writing out the
Fourier terms,
\begin{align}
  f = &f_{\rm transit,s} + \sum_{n=1}^{N} A_n \sin(n\omega_{\rm s}t)
  + \sum_{n=1}^{N} B_n \cos(n\omega_{\rm s}t)\\
  &+ \sum_{m=1}^{M} A_m \sin(m[\omega_{\rm \ell}t+\phi_{\rm \ell}])
  + \sum_{m=1}^{M} B_m \cos(m[\omega_{\rm \ell}t+\phi_{\rm \ell}]), \nonumber
\end{align}
for $N$ and $M$ the total number of harmonics at the short and long
periods, respectively, $A_i$ and $B_i$ the amplitudes for each
harmonic term (potentially negative), and $\omega_i = 2\pi / P_i$ the
angular frequency for $i$ the short or long period index.  We fixed
the ``phase-offset'' for the short period signal to be zero, and let
the reference time for the long period signal float by introducing
$\phi_{\rm \ell}$.  Since we did not a priori know how many harmonics
would be appropriate, we considered a number of different choices for
$N$ and $M$, and used the Bayesian information criterion to choose the
appropriate model (Table~\ref{tab:modelcompare}).

As an example, one possible model could be a transit, plus $N=2$
harmonics of sines and cosines at the short period, plus $M=1$
harmonics at the long period.  In this case, the free parameters would
be as follows.  For the transit, we would fit for the impact
parameter, the planet-to-star radius ratio, two quadratic limb
darkening parameters, the planet orbital period (equal to the short
period), the reference time for the transit, and the mean flux.  There
would be $2N=4$ additional Fourier amplitudes at the short period,
plus $2M=2$ Fourier amplitudes at the long period, and well as the
long period itself and its phase.  For this case, we therefore fitted
14 free parameters.

We implemented and fitted the models using \texttt{PyMC3}, which is
built on \texttt{theano}
\citep{salvatier_2016_PyMC3,exoplanet:theano}.  For the Fourier terms,
we used the default math operators.  For the exoplanet transit, we
used the model and derivatives implemented in \texttt{exoplanet}
\citep{exoplanet:exoplanet}.  Our priors are listed in
Table~\ref{tab:posterior}.  To speed up the fitting, we binned the
cleaned 2 minute lightcurves to 10 minute bins.  We correspondingly
scaled the uncertainties in the flux measurements by a factor of
$\sqrt{5}$.  Before sampling, we initialized each model to the maximum
a posteriori (MAP) solution.  We then sampled using \texttt{PyMC3}'s
gradient-based No-U-Turn Sampler \citep{hoffman_no-u-turn_2014}, and
used $\hat{R}$ as our convergence diagnostic
\citep{gelman_inference_1992}.
We tested our ability to successfully recover injected parameters
using synthetic data, before switching to the actual PTFO 8-8695
lightcurves.


\subsection{Fitting Results}

\startlongtable
\begin{deluxetable*}{lrrrrrrrr}
%
%\tabletypesize{\scriptsize}
%
\tablenum{1}
%
\tablecaption{Model Comparison.}
\label{tab:modelcompare}
%
\tablehead{
\colhead{Description} &
\colhead{$N_{\rm s}$} &
\colhead{$N_{\rm \ell}$} &
\colhead{$N_{\rm data}$} &
\colhead{$N_{\rm param}$} &
\colhead{$\chi^2$} &
\colhead{$\chi_{\rm red}^2$} &
\colhead{BIC} &
\colhead{$\Delta$BIC}
}
% pasted from
% /Users/luke/Dropbox/proj/billy/results/PTFO_8-8695_results/20200513_v0/bic_table_data.tex
% 
% Burnham and Anderson 2004.
% "Models having i ≤ 2 have substantial support (evidence), those in which 4 ≤
% i ≤ 7 have considerably less support, and models having i > 10 have
% essentially no support"
\startdata
Favored    & 3 &  2 &   2585 &      21 &  3102.4 &     1.210 &  3267.4 &     0.0 \\
% \hline
% Weakly favored &  
\hline
Disfavored  &  2 &  3 &   2585 &      21 &  3179.0 &     1.240 &  3344.0 &    76.6 \\
---         &  2 &  2 &   2585 &      19 &  3237.4 &     1.262 &  3386.7 &   119.3 \\
---         &  3 &  3 &   2585 &      23 &  3217.1 &     1.256 &  3397.9 &   130.4 \\
---         &  2 &  1 &   2585 &      17 &  3312.6 &     1.290 &  3446.1 &   178.7 \\
---         &  3 &  1 &   2585 &      19 &  3397.5 &     1.324 &  3546.8 &   279.4 \\
---         &  1 &  2 &   2585 &      17 &  4101.2 &     1.597 &  4234.8 &   967.3 \\
---         &  1 &  3 &   2585 &      19 &  4160.8 &     1.622 &  4310.1 &  1042.7 \\
---         &  1 &  1 &   2585 &      15 &  4318.4 &     1.680 &  4436.2 &  1168.8 \\
\enddata
%
\tablecomments{
	$N_{\rm s}$ and $N_{\rm \ell}$ are the number of harmonics at the short and long periods, respectively.
	$N_{\rm data}$ is the number of fitted flux measurements.
	$N_{\rm param}$ is the number of free parameters in the model.
	The Bayesian information criterion (BIC) and the difference from the maximum $\Delta {\rm BIC}$ are also listed.
}
\vspace{-1cm}
\end{deluxetable*}

%% Table of best fit parameters
\startlongtable
\begin{deluxetable*}{lrrr}
%
\tablecaption{ Best-fit radial velocity model parameters. }
\label{tab:posterior}
%
\tablenum{4}
%
\tablehead{
  \colhead{Parameter} & 
  \colhead{Credible Interval} & 
  \colhead{Maximum Likelihood} & 
  \colhead{Units}
}
\startdata
%\sidehead{~~~~~\it{Modified MCMC Step Parameters}}
%  $P_{b}$ & $1.338231466\pm 2.3e-08$ & $1.338231466$ & day \\
%  $T\rm{conj}_{b}$ & $2455804.515752^{+2.5e-05}_{-2.4e-05}$ & $2455804.515752$ & BJD$_{\rm TDB}$ \\
%  $e_{b}$ & $\equiv0.0$ & $\equiv0.0$ &  \\
%  $\omega_{b}$ & $\equiv0.0$ & $\equiv0.0$ & $^\circ$ \\
%  $K_{b}$ & $242.6^{+3.6}_{-3.5}$ & $242.6$ & m$\,{\rm s}^{-1}$ \\
%\hline
\sidehead{~~~~~\it{Orbital Parameters}}
  $P_{b}$ & $1.338231466\pm 2.3e-08$ & $1.338231466$ & day \\
  $T\rm{conj}_{b}$ & $2455804.515752^{+2.5e-05}_{-2.4e-05}$ & $2455804.515752$ & BJD$_{\rm TDB}$ \\
  $e_{b}$ & $\equiv0.0$ & $\equiv0.0$ &  \\
  $\omega_{b}$ & $\equiv0.0$ & $\equiv0.0$ & $^\circ$ \\
  $K_{b}$ & $242.6^{+3.6}_{-3.5}$ & $242.6$ & m$\,{\rm s}^{-1}$ \\
%\hline
\sidehead{~~~~~\it{Other Parameters}}
  $\gamma_{\rm HIRES}$ & $36.4^{+5.8}_{-5.9}$ & $36.4$ & m$\,{\rm s}^{-1}$ \\
  $\gamma_{\rm HARPS}$ & $-69.9^{+4.2}_{-4.1}$ & $-70.1$ & m$\,{\rm s}^{-1}$ \\
  $\gamma_{\rm CORALIE}$ & $-39.9^{+5.5}_{-5.2}$ & $-40.1$ & m$\,{\rm s}^{-1}$ \\
  $\dot{\gamma}$ & $-0.0422^{+0.0028}_{-0.0027}$ & $-0.0424$ & m$\,{\rm s}^{-1}\,{\rm day}^{-1}$ \\
  $\ddot{\gamma}$ & $\equiv0.0$ & $\equiv0.0$ &  \\
  $\sigma_{\rm HIRES}$ & $10.8^{+3.7}_{-2.7}$ & $8.2$ & $\rm m\,s^{-1}$ \\
  $\sigma_{\rm HARPS}$ & $13.0^{+3.7}_{-2.6}$ & $11.5$ & $\rm m\,s^{-1}$ \\
  $\sigma_{\rm CORALIE}$ & $13.8^{+6.6}_{-6.7}$ & $12.9$ & $\rm m\,s^{-1}$ \\
\enddata
%\tablenotetext{}{ 240000 links saved}
\tablenotetext{}{
  Reference epoch for $\gamma$,$\dot{\gamma}$,$\ddot{\gamma}$: 2455470 
}
\vspace{-2.5cm}
\end{deluxetable*}


We considered nine models, with the number of harmonics per frequency
$N$ and $M$ ranging from one to three.  To select our preferred model,
we used the Bayesian information criterion
(Table~\ref{tab:modelcompare}).  The model with the lowest BIC had
two harmonics at the short 11.74$\,$hr period, and two harmonics
at the long 11.96$\,$hr period.  All of the models have reduced
$\chi^2$ ranging between 1.37 and 1.51, which suggests a plausible,
though not perfect agreement between the data and models.

To explore where each model succeeded and failed, we split the raw
signal into its respective components (Figures~\ref{fig:splitsignal}
and~\ref{fig:splitsignalii}).  We also examined the phase-folded
signals (Figure~\ref{fig:phasefold}).  

In every model, the variability at the long period is a simple
sinusoid with peak-to-peak amplitude $\approx$10\%
(Figure~\ref{fig:phasefold}, top).  The variability at the short
period is always more complex.  A dip of depth $\approx$1.2\%, fit in
our model as a transit, lasts $\approx$0.75 hours.  Superposed on the
dip is a complex signal with peak-to-peak amplitude of about 4\%,
which peaks near phase 0.25, and reaches minimum brightness between
phases -0.5 and -0.25.

Outside of the primary dip, the short-period signal is relatively
smooth, at least from phases 0 to 0.5.  However the short-period
signal is asymmetric.  The flux from phases -0.5 to 0 shows what could
be a discontinuous jump, shortly after reaching minimum.  This jump
was visible in each of the nine models we considered, with different
choices for the number of harmonics.


 
% The main result that both models agree on: the extra power at the
% orbital frequency seems to not be ellipsoidal. Both of these models
% prefer to put the power at 1x the orbital frequency, not 2x the
% orbital frequency! This gives both Aorb0 and Borb0 non-zero, while
% Aorb1 and Borb1 seem to be more consistent with zero.




% The periodogram of the final residual (Figure~\ref{fig:splitsignal}
% bottom row) shows a weakly significant, poorly resolved peak at
% $\approx$8 days, consistent with the visual impression in the time
% domain that there could be a weak long-period signal present.


\subsection{Blend considerations}
\label{subsec:blend}

The TESS pixels are $\approx21$'' per side, and so we need to consider
whether light from neighboring stars could affect the photometry.  The
scene is shown in Figure~\ref{fig:scene}.  
The pixels used to
measure the background level are indicated with an `\texttt{X}' hatch,
and the pixels used for the final lightcurve are shown with the
`\texttt{/}' hatch.

The target star, PTFO
8-8605 (TIC 264461976), has a $T$-band magnitude of 14.0, and its position is shown with a
star.  
The other (unlabeled) star inside the target aperture, TIC 264461979, has $T=16.8$ and so cannot
contribute a signal with relative amplitude 10\%.
The only neighbor that is sufficiently close and bright that
its light might contaminate the target star is TIC 264461980, with
$T=14.8$, which we denote ``Star A''.  Star A is 23.6'' NW of our
target, and based on the magnitude difference could contribute up to
48\% the flux of our target star, PTFO 8-8695.  

Because PTFO 8-8695 has previously been identified to have periodicity
consistent with our measurement of $P_{\rm s}$, our main concern
regarding blending is the degree to which we can be certain that the
long-period signal at $P_{\rm \ell}$ also originates from PTFO 8-8695.
We took two approaches towards determining the source of the long-period signal.

First, we examined the CDIPS full frame image lightcurves of the
target, which are available on MAST \citep{bouma_cluster_2019}.
The maximal peak-to-peak beat amplitude is consistently $\approx$10\%
across apertures of radii 1, 1.5, and 2.25 pixels.
If Star A were the source of the long-period variability, we would expect the
peak variability amplitude to be smallest in the 1 pixel aperture, based on the
separation of the sources (Figure~\ref{fig:scene}, bottom).
From this test alone, it seems unlikely that Star A is the source of
the long-period signal.

Second, we examined the lightcurve of each pixel in the scene
individually.  We opted to use the
interactive tools implemented in
\texttt{lightkurve} \citep{lightkurve_2018}.  If Star A were the
source of the long-period variability, we would expect the pixels
nearest to Star A to show a sinusoidal signal with
amplitude exceeding $10\%$.  We find no evidence for
this being the case.  The pixel directly below Star A does not
clearly show the sinusoidal variability, and the peak-to-peak 
variability in that pixel is $\lesssim 8\%$.  In contrast, the
south-easternmost pixel within PTFO 8-8695's aperture (the pixel 
furthest from Star A that was used in the optimal aperture) shows the $P_{\rm \ell}$ sinusoidal
variability signal at $\approx 10\%$ amplitude.

As there is no evidence in favor of a blend scenario, we
conclude that both the $P_{\rm s}$ and $P_{\rm \ell}$ signals originate from PTFO 8-8695.

\section{Interpretation}
\label{sec:discussion}

The TESS dip does not phase up where it is supposed to...
Figure~\ref{fig:o_minus_c}

\section{Conclusions}
\label{sec:conclusions}



%%%%%%%%%%%%%%%%%%%%%%%%%%%%%%%%%%%%%%%%%%%%%%%%%%%%%%%%%%%%%%%%%%%%%%%%%%%%%%%

% \acknowledgements
% %
% This paper includes data collected by the TESS mission, which are
% publicly available from the Mikulski Archive for Space Telescopes
% (MAST).
% %
% Funding for the TESS mission is provided by NASA's Science Mission
% directorate.
% %
% This work made use of NASA's Astrophysics Data System Bibliographic
% Services.
% %
% Based on observations obtained at the Gemini Observatory, which is
% operated by the Association of Universities for Research in Astronomy,
% Inc., under a cooperative agreement with the NSF on behalf of the
% Gemini partnership: the National Science Foundation (United States),
% National Research Council (Canada), CONICYT (Chile), Ministerio de
% Ciencia, Tecnolog\'{i}a e Innovaci\'{o}n Productiva (Argentina),
% Minist\'{e}rio da Ci\^{e}ncia, Tecnologia e Inova\c{c}\~{a}o (Brazil),
% and Korea Astronomy and Space Science Institute (Republic of Korea).
% %
% Observations in the paper made use of the High-Resolution Imaging
% instrument Zorro at Gemini-South. Zorro was funded by the NASA
% Exoplanet Exploration Program and built at the NASA Ames Research
% Center by Steve B. Howell, Nic Scott, Elliott P. Horch, and Emmett
% Quigley.
% %
% This research has made use of the VizieR catalogue access tool, CDS,
% Strasbourg, France. The original description of the VizieR service was
% published in A\&AS 143, 23.
% %
% This work has made use of data from the European Space Agency (ESA)
% mission {\it Gaia} (\url{https://www.cosmos.esa.int/gaia}), processed
% by the {\it Gaia} Data Processing and Analysis Consortium (DPAC,
% \url{https://www.cosmos.esa.int/web/gaia/dpac/consortium}). Funding
% for the DPAC has been provided by national institutions, in particular
% the institutions participating in the {\it Gaia} Multilateral
% Agreement.
%
% (Some of) The data presented herein were obtained at the W. M. Keck
% Observatory, which is operated as a scientific partnership among the
% California Institute of Technology, the University of California and
% the National Aeronautics and Space Administration. The Observatory was
% made possible by the generous financial support of the W. M. Keck
% Foundation.
% The authors wish to recognize and acknowledge the very significant
% cultural role and reverence that the summit of Maunakea has always had
% within the indigenous Hawaiian community.  We are most fortunate to
% have the opportunity to conduct observations from this mountain.
%
% \newline
%

\software{
  \texttt{astrobase} \citep{bhatti_astrobase_2018},
  % \texttt{astroplan} \citep{astroplan2018},
  \texttt{astropy} \citep{astropy_2018},
  \texttt{astroquery} \citep{astroquery_2018},
  % \texttt{BATMAN} \citep{kreidberg_batman_2015},
  \texttt{corner} \citep{corner_2016},
  %\texttt{emcee} \citep{foreman-mackey_emcee_2013},
  \texttt{exoplanet} \citep{exoplanet:agol19}
  \texttt{exoplanet} \citep{exoplanet:exoplanet}, and its
  dependencies \citep{exoplanet:agol19, exoplanet:kipping13, exoplanet:luger18,
  	exoplanet:theano}.
  \texttt{IPython} \citep{perez_2007},
	\texttt{lightkurve} \citep{lightkurve_2018},
  \texttt{matplotlib} \citep{hunter_matplotlib_2007}, 
  \texttt{MESA} \citep{paxton_modules_2011,paxton_modules_2013,paxton_modules_2015}
  \texttt{numpy} \citep{walt_numpy_2011}, 
  \texttt{pandas} \citep{mckinney-proc-scipy-2010},
  \texttt{PyMC3} \citep{salvatier_2016_PyMC3},
  \texttt{radvel} \citep{fulton_radvel_2018},
  % \texttt{scikit-learn} \citep{scikit-learn},
  \texttt{scipy} \citep{jones_scipy_2001}.
}


% \facilities{
% 	{\it Astrometry}:
% 	Gaia \citep{gaia_collaboration_gaia_2016,gaia_collaboration_gaia_2018}.
% 	{\it Imaging}:
% 	Gemini:South~(Zorro; \citealt{scott_nessi_2018}.
% 	{\it Spectroscopy}:
% 	Keck:I~(HIRES; \citealt{vogt_hires_1994}),
% 	Euler1.2m~(CORALIE),
% 	ESO:3.6m~(HARPS; \citealt{mayor_setting_2003}).
% 	{\it Photometry}:
% 	CTIO:1.0m (Y4KCam),
% 	Danish 1.54m Telescope,
% 	El Sauce:0.356m,
% 	Elizabeth 1.0m at SAAO,
% 	Euler1.2m (EulerCam),
% 	Magellan:Baade (MagIC),
% 	Max Planck:2.2m	(GROND; \citealt{greiner_grond7-channel_2008})
% 	NTT,
% 	SOAR (SOI),
% 	TESS \citep{ricker_transiting_2015},
% 	TRAPPIST \citep{jehin_trappist_2011},
% 	VLT:Antu (FORS2).
% }

%
% The following are entries from Table 1 that are not otherwise cited
% in the text
%
% \nocite{wilson_wasp-4b_2008}
% \nocite{gillon_improved_2009}
% \nocite{winn_transit_2009}
% \nocite{hoyer_tramos_2013}
% \nocite{dragomir_terms_2011}
% \nocite{sanchis-ojeda_starspots_2011}
% \nocite{nikolov_wasp-4b_2012}
% \nocite{ranjan_atmospheric_2014}
% \nocite{huitson_gemini_2017}

% \input{WASP-4b_transit_time_table.tex}
% \input{WASP-4b_rv_table.tex}
% \input{model_fit_table.tex}
% \input{rv_model_posterior_table.tex}
% \input{pdot_table.tex}

%\clearpage
\bibliographystyle{yahapj}                            
\bibliography{bibliography} 


\listofchanges

\end{document}
